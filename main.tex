\documentclass[a4paper,dvipsnames, 11pt]{amsart} %pdf latex
\usepackage{preamble}

\begin{document}
\maketitle
\cite{FS19}
\begin{notation}
	We employ the following notations. 
	\begin{itemize}
		\item %
			For each category $\one{C}$,
			we write $\core{\one{C}}$ for the core groupoid of $\one{C}$.
		\item %
			For each groupoid $\one{G}$, we write $\pi_0(\one{G})$ for the set of connected components of $\one{G}$.
		\item %
			For each category $\one{C}$ and an object $c\in\one{C}$,
			we write $\one{C}_{c/}$ and $\one{C}_{/c}$ for the under category and the over category.
		\item %
			For each category $\one{C}$ and an object $c\in\one{C}$,
			the domain part defines a functor $\one{C}_{/c}\arr"\Sigma_c"\one{C}$,
			and we write $c^*$ and $\Pi_c$ for the right adjoints of $\Sigma_c$ and $c^*$ respectively,
			if exists.

			The dual of $\Sigma_c$, $c^*$, and $\Pi_c$ are
			denoted by $\arbSigma_c\colon\one{C}_{c/}\arr\one{C}$, $c_*\colon\one{C}\arr\one{C}_{c/}$, and $\arbPi_c\colon\one{C}\arr\one{C}_{c/}$.
		\item %
			We write $\Set$ for the catgeory of sets.
		\item %
			For each nonnegative integer $n$,
			we write $\bar{n}$ for the set $\{1,\ldots,n\}$.
		\item %
			We write $\FinSet$ for the catgeory defined as follows.
			\begin{itemize}
				\item %
					$\Obj(\FinSet)=\mathbb{N}$.
				\item %
					$\FinSet(n,m)=\Set(\bar{n},\bar{m})$.
				\item %
					$n\mapsto\bar{n}$ extends to a fully faithful functor $\FinSet\arr[hook]\Set$.
			\end{itemize}
		\item %
			For each category $\one{C}$ with finite coproducts,
			we write $-+-$ for the binary coporduct and $\emptyset$ for the initial object.

			We write ${\Atled}_c\colon c+c\arr c$ for the codiagonal.
	\end{itemize}
\end{notation}

\begin{definition}
	Suppose we are given
	a set $\Lambda\in\Set$.
	We define a category
	$\FS(\Lambda)$
	as the following Grothendieck construction.
	by the following pullback.
	\[
		\begin{tikzcd}[row sep=5mm]
			\FS(\Lambda)
			\ar[r]
			\ar[d]
			\pullback[rd]
				&
				\Set_{/\Lambda}
				\ar[d,"\Sigma_\Lambda"]
			\\
			\FinSet
			\ar[r,hook]
				&
				\Set
		\end{tikzcd}
	\]
	An element of $\FS(\Lambda)$ is called a
	\emph{$\lambda$-labelled finite set}.
\end{definition}
\begin{proposition}
	For each $\Lambda\in\Set$,
	$\FS(\Lambda)$ has finite colimits.
\end{proposition}
\begin{proof}
	Both $\FinSet\arr[hook]\Set$ and $\Set_{/\Lambda}\arr"\Sigma_\Lambda"\Set$ create finite colimits.
\end{proof}
\begin{definition}
	Let $\one{C}$ be a category with finite colimits.
	We write $\Cosp[\one{C}]$ for the category defined as follows.
	\begin{itemize}
		\item %
			$\Obj(\Cosp[\one{C}])=\Obj{\one{C}}$.
		\item %
			For each $c,c'\in\one{C}$,
			the homset is defined as
			\[
				\Cosp[\one{C}](c,c')=\pi_0\left(\core{(\one{C}_{c+c'/})}\right)
				\text{.}
			\]
		\item %
			The identity on $c$ is represented by the codiagonal $\Atled\colon c+c\arr c$.
		\item %
			The composition is obtained by applying $\pi_0(\core{(-)})$ to the following functor.
			\[
				\one{C}_{c+c'/}
				\times
				\one{C}_{c'+c''/}
				\arr"+"[][2]
				\one{C}_{c+c'+c'+c''/}
				\arr"{(c+\Atled_{c'}+c'')_*}"[][5]
				\one{C}_{c+c'+c''/}
				\arr"{\arbSigma_{\inc_{1,3}}}"[][3]
				\one{C}_{c+c''/}
			\]
		\qedhere %
	\end{itemize}
\end{definition}
\begin{definition}
	Let $\Lambda\in\Set$.
	We write $\Cospfn_\Lambda$ for $\Cosp[\FS(\Lambda)]$.
\end{definition}
\begin{proposition}
	Let $\one{C}$ be a category with finite colimits.
	Then the coproduct of $\one{C}$ lifts to a dagger compact monoidal structure on $\Cosp[\one{C}]$.
\end{proposition}
\begin{proof}
	Every object $c$ is self-dual by applying $\pi_0(\core{(-)})$ to the cacnonical isomorphism
	$\one{C}_{(c_0+c)+c_1}\cong\one{C}_{c_0+(c+c_1)}$.
	The involution
	$\dagger_{c,c'}\colon\Cosp[\one{C}](c,c')\arr\Cosp[\one{C}](c',c)$
	is obtained by applying $\pi_0(\core{(-)})$ to the canonical isomorphism
	$\one{C}_{c+c'/}\cong\one{C}_{c'+c/}$ for any $c,c'\in\one{C}$.
\end{proof}
\begin{definition}
\end{definition}

\end{document}
